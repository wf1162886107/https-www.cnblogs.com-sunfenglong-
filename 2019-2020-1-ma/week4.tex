\documentclass[UTF8,oneside,12pt]{article}  %--- 单面打印用oneside,双面打印就改为twoside
                                                                        %--- 12pt对应中文字号小四号
\usepackage{ctex}
%--- 可以用下面一命令代替上面两行 -----------
%\documentclass[UTF8,oneside,12pt]{ctexart}   %-------ctex + article=ctexart
%-----------------------------------------------


%------- 一些常用数学宏包 --------------------------
\usepackage{amsmath,amsthm}
\usepackage{amsfonts,amssymb}
\usepackage{mathrsfs}
\usepackage{bm}
\usepackage{dsfont}
%------------------------------------------------------
\usepackage[colorlinks,linkcolor=red,anchorcolor=blue,citecolor=red]{hyperref}  %--- 超链接宏包
\usepackage{cite}          %--- 引用宏包

\usepackage{indentfirst}   %-- 首行缩进宏包
\usepackage{array,tabularx}           %-- 列表宏包
\usepackage{diagbox}      %-- 改宏包可用于制做有斜线表头的表格
\usepackage{color}          %-- 颜色宏包
\usepackage{graphicx}    %--- 插图宏包

\usepackage{caption}      %--- 该宏包用来定制图表标题
\captionsetup{font={footnotesize,bf}}  %-- 图表标题设置为小五号黑体


%-------  geometry 宏包用来调整页面的页边距。默认的页面边距为A4标准。
%\usepackage{geometry}

%------  根据学校文件中的课程论文模板的实际情况,下面通过指定geometry 宏包的选项参数来自定义页面边距 ----------------
\usepackage[
  hoffset=5mm,%%装订线
  left=2.5cm,      %% 左边距
  right=2.5cm,     %% 右边距
  top=2.5cm,       %% 上边距 包括页眉
  bottom=2cm,    %% 下边距 包括页脚
  headheight=0cm, %%页眉
  footskip=1.0cm,  %% 页脚
  paperwidth=21cm,
  paperheight=29.7cm
  ]{geometry}

\def\baselinestretch{1}         %-- 行间距为1倍行间距
%\setlength{\baselineskip}{20pt}  %-- 行间距为固定值20磅
%\setlength{\parskip}{0.7ex plus0.3ex minus0.3ex} % 段落间距
\allowdisplaybreaks[4]       %% 允许多行公式中间换页
%--------------------------------



%----------------------- 字体设置 ----------------------------------------
\usepackage{fontspec}
%\usepackage{xeCJK}             %--中文字体
\usepackage{newtxtext}       %-- 设置文字部分的英文字体为 Times New Roman
%\usepackage{newtxmath}    %-- 设置数学式中的英文字体为 Times New Roman
%-----------------------------------------------
%\setCJKmonofont{FZYTK.ttf}  %--方正姚体
%\newcommand{\fangzhengyaoti}{\CJKfamily{FZYTK.ttf}}
 \newcommand{\myfont}[1]{\setCJKfamilyfont{font}{#1}\CJKfamily{font}}


%\newcommand{\chuhao}{\fontsize{42pt}{\baselineskip}\selectfont}
\newcommand{\xiaochuhao}{\fontsize{36pt}{\baselineskip}\selectfont}
%\newcommand{\yihao}{\fontsize{28pt}{\baselineskip}\selectfont}
\newcommand{\xiaoyihao}{\fontsize{24pt}{\baselineskip}\selectfont}
\newcommand{\erhao}{\fontsize{21pt}{\baselineskip}\selectfont}
%\newcommand{\xiaoerhao}{\fontsize{18pt}{\baselineskip}\selectfont}
\newcommand{\sanhao}{\fontsize{15.75pt}{\baselineskip}\selectfont}
\newcommand{\sihao}{\fontsize{14pt}{\baselineskip}\selectfont}
\newcommand{\xiaosihao}{\fontsize{12pt}{\baselineskip}\selectfont}
\newcommand{\wuhao}{\fontsize{10.5pt}{\baselineskip}\selectfont}
%\newcommand{\xiaowuhao}{\fontsize{9pt}{\baselineskip}\selectfont}
%\newcommand{\liuhao}{\fontsize{7.875pt}{\baselineskip}\selectfont}
%\newcommand{\qihao}{\fontsize{5.25pt}{\baselineskip}\selectfont}
%-------------------------------------------------------------------------

%---------------- 设置目录、标题 -------------------------------------
\usepackage{titletoc,titlesec}

\renewcommand{\contentsname}{{
\begin{center}
\sihao\heiti 目\ \ 录
\end{center}
}}
\usepackage[titles]{tocloft}
\renewcommand\cftsecdotsep{\cftdotsep}
\renewcommand\cftsecleader{\cftdotfill{\cftsecdotsep}}

\newcommand{\sectionname}{节}
\titleformat{\section}[block]{\sihao\fangsong}{\thesection}{10pt}{}
\titleformat{\subsection}[block]{\xiaosihao\heiti}{\thesubsection}{10pt}{}
\titleformat{\subsubsection}[block]{\xiaosihao\fangsong}{\thesubsubsection}{10pt}{}
%-- 下面设置各级标题左边以及与上下文内容的间距 ---
\titlespacing{\section}{0pt}{0.6ex}{0.6ex}
\titlespacing{\subsection}{0.5ex}{0.3ex}{0.3ex}
\titlespacing{\subsubsection}{1ex}{0.3ex}{0.3ex}
%------------------------------------------------------------------------



%--------------- 设置定理环境 -----------------------------------------
\newtheoremstyle{DingLi1}
        {0.6ex plus 0.5ex minus .2ex}%上方的空行
        {0.6ex plus 0.5ex minus .2ex}%下方的空行
        {\kaishu}%内容字体
        {}%缩进
        {\heiti}%定理头部字体
        {.}%定理头部后的标点
        {1em}%定理头部后的空格
        {}%定理头部的说明
\theoremstyle{DingLi1}
\numberwithin{equation}{section}
\newtheorem{theorem}{\hspace{2em}定理}[section]
\newtheorem{definition}{\hskip 2em 定义}[section]
\newtheorem{lemma}{\hskip 2em 引理}[section]
\newtheorem{corollary}{\hskip 2em 推论}[section]
\newtheorem{assumption}{\hskip 2em 条件}[section]
\newtheorem{remark}{\hskip 2em 注}[section]
\newtheorem{proposition}{\hskip 2em 命题}[section]
\newtheorem{property}{\hskip 2em 性质}[section]

\newtheoremstyle{DingLi2}
        {1ex plus 0.5ex minus .2ex}%上方的空行
        {1ex plus 0.5ex minus .2ex}%下方的空行
        {\songti}%内容字体
        {}%缩进
        {\heiti\large\color{red} }%定理头部字体
        {.}%定理头部后的标点
        {1em}%定理头部后的空格
        {}%定理头部的说明
\theoremstyle{DingLi2}
\newtheorem{example}{\hskip 2em 问题}[section]
%---------------------------------------------------------------------

%---------- 设置证明环境 ---------------------------------------------
\makeatletter% amsthm: get rid of \itshape and \@addpunct{.}
\renewenvironment{proof}[1][\proofname]{\par%
\pushQED{\qed}%
\normalfont \topsep6\p@\@plus6\p@\relax%
\trivlist%
\item[\hskip\labelsep%
#1]\ignorespaces%
}{%
\popQED\endtrivlist\@endpefalse%
}
\makeatother
\renewcommand{\proofname}{\heiti\large\color{blue} 证明}%注意字体的设置
%---------------------------------------------------------------------

%-------------- 自定义命令 ------------------------------
\def\d{\mathop{}\!\mathrm{d}} %-- 一阶微分符号d
\def\l{\lambda}\def\L{\Lambda}
\def\D{\Delta}
\def\de{\delta}
\def\gm{\gamma}
\def\a{\alpha}
\def\b{\beta}
\def\div{{\rm div}}
%---------------------------------------------------------

\begin{document}



\title{数学分析作业}
\author{}
\date{}
\maketitle

\section{第4周}



\begin{example}
设$A,B$为非空有界数集, 并且$A\subset B$, 证明
   $$\inf B\leq \inf A\leq \sup A\leq \sup B.$$
\end{example}
\begin{proof}
显然, $\inf A\leq \inf B$. 下证$\inf B\leq \inf A$并且$\sup A\leq \sup B$.

假设$\inf B>\inf A$, 则$\inf B$不是集合$A$的下界, 从而存在$x_0\in A$使得$\inf B>x_0$. 另一方面, 由于$A\subset B$,从而也有$x_0\in B$, 于是
$$\inf B>x_0\geq \inf B,$$
矛盾.

假设$\sup A>\sup B$, 则$\sup B$不是集合$A$的上界, 从而存在$x_1\in A$使得$\sup B<x_1$. 另一方面, 由于$A\subset B$,从而也有$x_1\in B$, 于是
$$\sup B<x_1\leq \sup B,$$
矛盾.

综上, $\inf B\leq \inf A\leq \sup A\leq \sup B.$
\end{proof}

\begin{example}
设$S$为非空有下界(不一定有上界)的数集, 并且$\inf S>0$, 证明集合
   $$S^{-1}=\left\{x^{-1}\ |\ x\in S \right\}$$
   有界并且
   $$\sup S^{-1}>0,\quad\inf S^{-1}\geq 0,\quad \sup S^{-1}=\frac{1}{\inf S}.$$
\end{example}
\begin{proof}
Step1. 任取$y\in S^{-1}$, 令$x=y^{-1}$, 则$x\in S$,
$$x\geq \inf S>0,$$
从而
\begin{equation}\label{4.1-1}
0<y\leq \frac{1}{\inf S},\quad \forall y\in S^{-1}.
\end{equation}
所以$0$是$S^{-1}$的一个下界, $\frac{1}{\inf S}$是$S$的一个上界, 并且
$$\sup S^{-1}>0,\quad\inf S^{-1}\geq 0.$$

Step2. 由Step1可知$\sup S^{-1}\leq \frac{1}{\inf S}$. 下面排除$\sup S^{-1}< \frac{1}{\inf S}$的情况.

反证法, 假设$\sup S^{-1}< \frac{1}{\inf S}$成立, 由于$\sup S^{-1}>0$, 则
$$0<\inf S<\frac{1}{\sup S^{-1}}.$$
所以$\frac{1}{\sup S^{-1}}$不是$S$的下界, 存在$x\in S$使得
$$0<\inf S\leq x<\frac{1}{\sup S^{-1}}.$$
令$y=x^{-1}$, 则$y\in S^{-1}$,
$$0<\frac{1}{y}=x<\frac{1}{\sup S^{-1}},$$
从而
$$y>\sup S^{-1}.$$
\end{proof}

\begin{example}
证明以下等式和不等式:
\begin{itemize}
    \item[(1)] 设$a,b\in \Bbb{R}$, 则
    $$\big||a|-|b|\big|\leq |a-b|,\quad \frac{|a+b|}{1+|a+b|}\leq \frac{|a|}{1+|a|}+\frac{|b|}{1+|b|}.$$

    \item[(2)] 设$a,b\in \Bbb{R}$, $n\in \Bbb{N}_+$且$n\geq 2$, 则
    $$\begin{array}{l}
    a^n-b^n=(a-b)(a^{n-1}+a^{n-2}b+\cdots+ab^{n-2}+b^{n-1}),
    \end{array}$$
    \item[(3)] Bernoulli(伯努利)不等式: 设$h\geq -1$, $n\in \Bbb{N}_+$, 则
        $$(1+h)^n\geq 1+nh.$$
        特别地, 如果还有$h\geq 0$并且$n>2$, 则还成立
        $$(1+h)^n>\frac{n(n-1)}{2}h^2>\frac{n^2h^2}{4}.$$

    \item[(4)] 算术-几何平均值不等式: 设$a_1,a_2,\cdots,a_n$是$n$个非负实数, 则
    $$\frac{a_1+a_2+\cdots+a_n}{n}\geq \sqrt[n]{a_1\cdot a_2\cdot\cdots\cdot a_n}.$$
    如果$a_1,a_2,\cdots,a_n$都是正实数, 还成立几何-调和平均值不等式:
    $$\sqrt[n]{a_1\cdot a_2\cdot\cdots\cdot a_n}\geq \frac{n}{\frac{1}{a_1}+\frac{1}{a_2}+\cdots+\frac{1}{a_n}}.$$
    \item[(5)] $$\begin{array}{l}
    \sum\limits_{k=1}^n k^2=1^2+2^2+3^2+\cdots n^2=\frac{n(n+2)(2n+1)}{6}.\\
    \sum\limits_{k=1}^n k^3=1^3+2^3+3^3+\cdots n^3=\left(\frac{n(n+1)}{2} \right)^2
    \end{array}$$

    \textbf{注:} $\sum\limits_{k=1}^n s_k$表示对$s_1,s_2,\cdots,s_n$按下标$k$从$1$到$n$求和, 即$\sum\limits_{k=1}^n s_k=s_1+s_2+\cdots+s_n$.

    \item[(6)] 若$0<x<\frac{\pi}{2}$, 则$\sin x<x <\tan x$. 若$x\in \Bbb{R}$, 则$|\sin x|\leq |x|$.
    \end{itemize}
\end{example}


\end{document}