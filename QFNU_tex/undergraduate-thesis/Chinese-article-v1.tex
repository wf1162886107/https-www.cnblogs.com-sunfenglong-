\documentclass[UTF8,oneside,12pt]{article}
\usepackage{ctex}
%--- 可以用下面一命令代替上面两行 -----------
%\documentclass[10pt]{ctexart}   %-------ctex + article, 正文字体大小规定为10pt --------
%-----------------------------------------------


%------- 一些常用数学宏包 --------------------------
\usepackage{amsmath,amsthm}
\usepackage{amsfonts,amssymb}
\usepackage{mathrsfs}
\usepackage{bm}
\usepackage{dsfont}
%------------------------------------------------------
\usepackage[colorlinks,linkcolor=red,anchorcolor=blue,citecolor=red]{hyperref}  %--- 超链接宏包
\usepackage{cite}          %--- 引用宏包

\usepackage{indentfirst}     %-- 首行缩进宏包
\usepackage{array}
\usepackage{diagbox}

\usepackage{color}

\usepackage{graphicx}   %--- 插图宏包
\usepackage{caption}
\captionsetup{font={footnotesize,bf}}


%-------  geometry 宏包用来调整页面的页边距。默认的页面边距为A4标准。
%\usepackage{geometry}

%------  下面通过指定geometry 宏包的选项参数来自定义页面边距 ----------------
\usepackage[
  hoffset=5mm,%%装订线
  left=2.5cm,      %% 左边距
  right=2.5cm,     %% 右边距
  top=2.5cm,       %% 上边距 包括页眉
  bottom=2cm,    %% 下边距 包括页脚
  headheight=0cm, %%页眉
  footskip=1.0cm,  %% 页脚
  paperwidth=21cm,
  paperheight=29.7cm
  ]{geometry}

\def\baselinestretch{1}         %-- 行间距为1倍行间距
%\setlength{\baselineskip}{20pt}  %-- 行间距为固定值20磅
%\setlength{\parskip}{0.7ex plus0.3ex minus0.3ex} % 段落间距
\allowdisplaybreaks[4]       %% 允许多行公式中间换页



%--------------------------------



%----------------------- 字体设置 ----------------------------------------

%---- 英文字体设置为 Times New Roman -------
\usepackage{fontspec}
\usepackage{newtxtext}       %-- 设置文字部分的英文字体为 Times New Roman
%\usepackage{newtxmath}    %-- 设置数学式中的英文字体为 Times New Roman
%-----------------------------------------------

%\newcommand{\chuhao}{\fontsize{42pt}{\baselineskip}\selectfont}
%\newcommand{\xiaochuhao}{\fontsize{36pt}{\baselineskip}\selectfont}
%\newcommand{\yihao}{\fontsize{28pt}{\baselineskip}\selectfont}
%\newcommand{\xiaoyihao}{\fontsize{24pt}{\baselineskip}\selectfont}
%\newcommand{\erhao}{\fontsize{21pt}{\baselineskip}\selectfont}
%\newcommand{\xiaoerhao}{\fontsize{18pt}{\baselineskip}\selectfont}
\newcommand{\sanhao}{\fontsize{15.75pt}{\baselineskip}\selectfont}
\newcommand{\sihao}{\fontsize{14pt}{\baselineskip}\selectfont}
\newcommand{\xiaosihao}{\fontsize{12pt}{\baselineskip}\selectfont}
\newcommand{\wuhao}{\fontsize{10.5pt}{\baselineskip}\selectfont}
%\newcommand{\xiaowuhao}{\fontsize{9pt}{\baselineskip}\selectfont}
%\newcommand{\liuhao}{\fontsize{7.875pt}{\baselineskip}\selectfont}
%\newcommand{\qihao}{\fontsize{5.25pt}{\baselineskip}\selectfont}

%-------------------------------------------------------------------------

%---------------- 设置目录、标题 -------------------------------------
\usepackage{titletoc,titlesec}
[block]
\renewcommand{\contentsname}{{
\begin{center}
\sihao\heiti 目\ \ 录
\end{center}
}}
\newcommand{\sectionname}{节}
\titleformat{\section}[block]{\sihao\fangsong}{\thesection}{10pt}{}
\titleformat{\subsection}[block]{\xiaosihao\heiti}{\thesubsection}{10pt}{}
\titleformat{\subsubsection}[block]{\xiaosihao\fangsong}{\thesubsubsection}{10pt}{}

\titlespacing{\section}{0pt}{0.6ex}{0.6ex}
\titlespacing{\subsection}{0.5ex}{0ex}{0ex}
\titlespacing{\subsubsection}{1ex}{0ex}{0ex}

%--------------- 设置定理环境 -----------------------------------------
\newtheoremstyle{DingLi}
  {1.5ex plus 1ex minus .2ex}%上方的空行
        {1.5ex plus 1ex minus .2ex }%下方的空行
        {\kaishu}%内容字体
        {}%缩进
        {\heiti}%定理头部字体
        {.}%定理头部后的标点
        {1em}%定理头部后的空格
        {}%定理头部的说明
\theoremstyle{DingLi}

\numberwithin{equation}{section}


\newtheorem{theorem}{\hspace{2em}定理}
%\newtheorem{theorem}{\hspace{2em}定理}
\newtheorem{definition}{\hskip 2em 定义}
\newtheorem{lemma}{\hskip 2em 引理}
\newtheorem{corollary}{\hskip 2em 推论}
\newtheorem{assumption}{\hskip 2em 条件}
\newtheorem{remark}{\hskip 2em 注}
\newtheorem{proposition}{\hskip 2em 命题}
\newtheorem{property}{\hskip 2em 性质}
\newtheorem{example}{\hskip 2em 例}
%---------------------------------------------------------------------



%---------- 设置证明环境 ---------------------------------------------
\makeatletter% amsthm: get rid of \itshape and \@addpunct{.}
\renewenvironment{proof}[1][\proofname]{\par%
\pushQED{\qed}%
\normalfont \topsep6\p@\@plus6\p@\relax%
\trivlist%
\item[\hskip\labelsep%
#1]\ignorespaces%
}{%
\popQED\endtrivlist\@endpefalse%
}
\makeatother
\renewcommand{\proofname}{\heiti{证明}}%注意字体的设置


%---------------------------------------------------------------------


%-------------- 自定义命令 ------------------------------
\def\l{\lambda}\def\L{\Lambda}
\def\D{\Delta}
\def\de{\delta}
\def\d{\mathop{}\!\mathrm{d}}
\def\gm{\gamma}
\def\a{\alpha}
\def\b{\beta}

\def\div{{\rm div}}


\begin{document}

\vskip 2cm

\tableofcontents
\thispagestyle{empty}
\newpage

\pagestyle{plain}
\setcounter{page}{1}


\noindent\framebox[4.45cm][l]{\shortstack[l]{\linespread{2.0}\wuhao 成 \hspace{1.5em} 绩:{\rule{20mm}{0.4pt}}   \\\linespread{2.0} \quad \\ \wuhao 指导教师:{\rule{20mm}{0.4pt}}}}
\vskip 1.5cm
\begin{center}
\sanhao \heiti 这里是论文标题
\end{center}
\setlength{\parskip}{1ex plus 0.5ex minus 0.2ex}
{\fangsong
\begin{center}
数学与应用数学(师范)专业学生  \quad  张三
\end{center}
\begin{center}
指导教师  \quad  李四
\end{center}
}
%-----------  中英文摘要 -----------------------------------------------
{\wuhao
\addcontentsline{toc}{section}{摘要} %---- 将“摘要”加入到目录中 -----

\noindent {\heiti 摘要}:
摘要主要是说明研究工作的目的、方法、结果和结论. 摘要应具有独立性和自含性, 即不阅读全文, 就能获得论文必要的信息, 使读者确定有无必要阅读全文. 摘要中应用第三人称的方法记述论文的性质和主题, 不使用``本文''、``作者''等作为主语,应采用``对…进行了研究''、``报告了…现状''、“进行了…调查”等表达方式。排除在本学科领域已成为常识的内容,不得重复题目中已有的信息。语句要合乎逻辑关系,尽量同正文的文体保持一致。结构要严谨,表达要简明,语义要确切,一般不再分段落。对某些缩略语、简称、代号等,除了相邻专业的读者也能清楚理解的以外,在首次出现处必须加以说明。摘要中通常不用图表、化学结构式以及非公知公用的符号和术语。摘要包含中文摘要和外文摘要. 中文摘要字数约为200-300字, 外文摘要约为200-300个实词。



\addcontentsline{toc}{section}{关键词}
\noindent {\heiti 关键词}:
 甲\   乙\   丙\  丁\   戊
}


\begin{center}
{\sanhao English title}
\end{center}
\begin{center}
Student majoring in applied mathematics\quad\quad     Long Aotian
\end{center}
\begin{center}
Tutor\quad\quad     Zhang San
\end{center}

{\wuhao
\addcontentsline{toc}{section}{Abstract}
\noindent{\bf Abstract}:
Blabla ..........

\addcontentsline{toc}{section}{Key words}
\noindent{\bf Key words}:
Aa;  Bb;  Cc;  Dd;  Ee

}
%-------------- 以下为正文部分 -----------------------------
\section{引言}

前言(引言)主要说明研究工作的目的、范围,对前人的研究状况进行评述分析,阐明研究设想、研究方法、实验设计、预期结果、成果的意义等。

在文献\cite{Liang-Xu-Auto18}(\textcolor[rgb]{1.00,0.00,0.00}{注意这里对参考文献的交叉引用方法})中, 作者研究了.......
%--------  在“参考文献”部分,给这条文献加了一标签 Liang-Xu-Auto18


\section{一个一级标题,自动编号}

\subsection{一个二级标题,自动编号}

\subsubsection{一个三级标题,自动编号}


下面是一个定义. 将对应的\LaTeX{}环境命令里的definition换成theorem, lemma, proposition, corollary,example, remark,就得到定理、引理、命题、推论、例、注等)
\begin{definition}[\cite{Liang-Xu-Auto18}]\label{Def: positive def matrix}
%------\begin{definition}[引用的定义所在的参考文献的标签] \label{给这个定义设置一个标签}
%----- 标签lable里的Def: positive def matrix是作者自己加上去的,为了方便交叉引用,加的标签最好有一定的意义,容易记忆
$n$阶实对称矩阵$A$为正定的,如果它所对应的二次型$X^T A X$是正定的,即对任意非零的$n$维列向量$X$, 有$X^T A X>0$.
\end{definition}



根据定义\ref{Def: positive def matrix}(\textcolor[rgb]{1.00,0.00,0.00}{注意这里的交叉引用方法}), 我们有......

下面是性质,还有一个列表的使用例子,注意列表编号的格式。
\begin{property}\label{Property: positive matrix}
如果$A$和$B$都是正定矩阵, 则有:
\begin{itemize}
  \item[(1)]$A+B$是正定矩阵;
  \item[(ii)]$kA$$(k>0)$是正定矩阵;
  \item[(bla)]blablabla;
  \item[1.]
  \item[i.]
  \item[A.]
\end{itemize}
\end{property}

以下是一个引理。
\begin{lemma}
设$E:\Bbb{R}^+ \to \Bbb{R}^+$ ($\Bbb{R}^+=[0,+\infty)$)是一个单调递减的函数且存在常数$T>0$,使得
$$\int_t^{\infty} E(s) \d s \leq TE(t),\quad \forall t\in \Bbb{R}^+,$$
则
$$E(t)\leq E(0) e^{1- \frac{t}{T}},\quad \forall t\geq T. $$
\end{lemma}

下面是一个定理及证明, 注意不等式(\ref{ineq-1})的交叉引用方法.
\begin{theorem}
设$E$是定义在$[0,\infty)$上的非负递减函数. 如果
\begin{equation}\label{ineq-1}  %----- 标签lable里的ineq-1是作者自己加上去的,为了方便交叉引用,加的标签最好有一定的意义,容易记忆
\int_S^{\infty} E(t) \d t \leq CE(S),\quad \forall S\geq S_0,
\end{equation}
其中$S_0$, $C$为固定常数, 则
$$E(t)\leq E(0)e^{1-\frac{t}{S_0+C}},\quad \forall t\geq 0.$$
\end{theorem}

\begin{proof}
若$0\leq S\leq S_0$, 由(\ref{ineq-1})式可知
\begin{eqnarray*}
% \nonumber to remove numbering (before each equation)
  \int_S^{\infty} E(t)\d t &=& \int_S^{S_0} E(t)\d t +\int_{S_0}^{\infty} E(t)\d t\\
  &\leq &(S_0-S) E(S)+CE(S_0)\\
 &= &S_0 E(S) +CE(S)
\end{eqnarray*}
因此, 对$\forall S\geq 0$, 有
$$\int_S^{\infty} E(t) \d t \leq  (S_0+C)E(S).$$
由引理3得
$$E(t)\leq E(0)e^{1-\frac{t}{S_0+C}},\quad \forall t\geq 0.$$
\end{proof}
\begin{remark}
这里是一个注。
\end{remark}

\begin{theorem}[局部存在性与唯一性, \cite{Liang-Xu-Auto18}]  \label{Thm: local existence}
 假设\textbf{条件} 成立, 则存在依赖于初始二次能量~$\mathscr{E}(0)$ 的~$T>0$ 使得问题在时间区间~$(-\infty, T]$ 上有弱解. 另外, 我们有下面的能量恒等式成立:
\begin{eqnarray}
&&\mathscr{E}+\int_0^t\int_\Omega |u_t|^{m+1} \d x\d \tau-\frac12 \int_0^t\int_0^{-\infty}|\nabla w(\tau,s)|_2^2 \mu'(s)\d s\d \tau\nonumber\\
&&=\mathscr{E}(0)+\int_0^t\int_\Omega |u|^{p-1}uu_t\d x\d\tau,\label{4 quadratic energy identity}
\end{eqnarray}
\end{theorem}

\subsection{数学公式、符号的例子}

行列式的例子
\begin{equation*}
  |\lambda E- A|=
  \begin{vmatrix}
   \lambda-a_{11} & -a_{12} & -a_{13}&\cdots &-a_{1n} \\
    -a_{21} & \lambda-a_{22} & -a_{23} &\cdots & -a_{2n}\\
    \vdots & \vdots & \vdots&\ddots&\vdots \\
    -a_{n1} & -a_{n2} & -a_{n3} &\cdots& \lambda -a_{nn}
 \end{vmatrix}
\end{equation*}

矩阵的例子
\begin{equation*}
A=(a_{ij})_{n\times n} =
\begin{pmatrix}
  a_{11} & a_{12} & a_{13} & \cdots & a_{1n}\\
  a_{21} & a_{22} & a_{23} &\cdots & a_{2n}\\
 \vdots & \vdots & \vdots & \ddots& \vdots\\
 a_{n1} & a_{n2} & a_{n3} &\cdots & a_{nn}
\end{pmatrix}
\end{equation*}

方程组的例子
\begin{equation*}
\left\{   %------ 定界符是大括号{。 \left\{   与后面的\right. 对应
    \begin{array}{l} %---  左对齐(l=left)
      u_{tt}-\Delta u+|u_t|^{m-1}u_t=|u|^{p-1}u,\quad (x,t)\in \Bbb{R}^n\times (0,\infty),\\
      u(0,x)=u_0(x),\quad u_t(x,0)=u_1(x),
      \end{array}
\right.
\end{equation*}

\begin{equation*}
\left\{
   \begin{array}{rl} %---- 分为两列,第一列右对齐(r=right),第二列左对齐(l=left)
      -x & \text{if } x < 0,\\
       0 & \text{if } x = 0,\\
       x & \text{if } x > 0.
\end{array} 
\right.
\end{equation*}


长公式
\begin{eqnarray*}
J(\psi_t(v);t)&=&\frac{p-2}{2p}(|\nabla \psi_t(v)|_2^2+b|\psi_t(v)|_2^2)+\frac1p I(\psi_t(v);t) \\
                &=&\frac{p-2}{2p}s^2(v;t)\|v\|^2 \\
                &=&\frac{p-2}{2p}(k(t))^{-\frac{2}{p-2}}\|v\|^{\frac{2p}{p-2}}.
\end{eqnarray*}
\begin{eqnarray*}
  &       &\frac{\gamma_a^p\left(2\rho(0)\right)^{1-\frac{p}{2}}}{\left(p-2\right)k(T_3)}\leq T^* \\
  &\leq & T_3:= \frac{8(p-1)(a\l_1+1)\rho(0)}{(p-2)^2[(p-2)(b+\l_1)\rho(0)-p(a\l_1+1)J(u_0;0)]};
  \end{eqnarray*}
  
  
 一个具有斜线表头的表格
 \begin{center}
\begin{tabular}{|c|c|c|} %------ 三列的列表,每列都居中对齐
  \hline
  % after \\: \hline or \cline{col1-col2} \cline{col3-col4} ...
  \diagbox{$X$}{$Y$} & $a$ & $b$ \\
  \hline
    $c$ & 1   &0 \\
   \hline
  $d$ & 0 &1 \\
  \hline
\end{tabular}
 \end{center}

  
\section{又一个一级标题 }

下面是一个例
\begin{example}
  这是一个例子
\end{example}




%-------------------------------------
\addcontentsline{toc}{section}{致谢} %---- 加入到目录中 -----
\section*{\heiti\xiaosihao 致谢}

本文的写作过程中,得到了李四老师的悉心指导与修改, 在此表示感谢.

\addcontentsline{toc}{section}{参考文献}         %---- 将“参考文献”加入到目录中 ---
\renewcommand\refname{\heiti \xiaosihao 参考文献}
\begin{thebibliography}{100}
\wuhao
\bibitem{jiang-06}
姜国. 正定矩阵的判定及性质[J]. 湖北师范学院学报(自然科学版), 2006(01): 97-100.

\bibitem{LiLiqun-17}
李立群. 正定矩阵及其应用[J]. 山东农业工程学院学报, 2017, 34(07): 28-30.

\bibitem{Liang-Xu-Auto18}  
Xiao Liang, JuanJuan Xu. Control for networked control systems with remote and local controllers over unreliable communication channel[J]. Automatica, 2018, 98(2018): 86-94.


\end{thebibliography}






\end{document} 