\documentclass[UTF8,oneside,12pt]{article}  %--- 单面打印用oneside,双面打印就改为twoside
                                                                        %--- 12pt对应中文字号小四号
\usepackage{ctex}
%--- 可以用下面一命令代替上面两行 -----------
%\documentclass[UTF8,oneside,12pt]{ctexart}   %-------ctex + article=ctexart
%-----------------------------------------------


%------- 一些常用数学宏包 --------------------------
\usepackage{amsmath,amsthm}
\usepackage{amsfonts,amssymb}
\usepackage{mathrsfs}
\usepackage{bm}
\usepackage{dsfont}
%------------------------------------------------------
\usepackage[colorlinks,linkcolor=red,anchorcolor=blue,citecolor=red]{hyperref}  %--- 超链接宏包
\usepackage{cite}          %--- 引用宏包

\usepackage{indentfirst}   %-- 首行缩进宏包
\usepackage{array,tabularx}           %-- 列表宏包
\usepackage{diagbox}      %-- 改宏包可用于制做有斜线表头的表格
\usepackage{color}          %-- 颜色宏包
\usepackage{graphicx}    %--- 插图宏包

\usepackage{caption}      %--- 该宏包用来定制图表标题
\captionsetup{font={footnotesize,bf}}  %-- 图表标题设置为小五号黑体


%-------  geometry 宏包用来调整页面的页边距。默认的页面边距为A4标准。
%\usepackage{geometry}

%------  根据学校文件中的课程论文模板的实际情况,下面通过指定geometry 宏包的选项参数来自定义页面边距 ----------------
\usepackage[
  hoffset=5mm,%%装订线
  left=2.5cm,      %% 左边距
  right=2.5cm,     %% 右边距
  top=2.5cm,       %% 上边距 包括页眉
  bottom=2cm,    %% 下边距 包括页脚
  headheight=0cm, %%页眉
  footskip=1.0cm,  %% 页脚
  paperwidth=21cm,
  paperheight=29.7cm
  ]{geometry}

\def\baselinestretch{1}         %-- 行间距为1倍行间距
%\setlength{\baselineskip}{20pt}  %-- 行间距为固定值20磅
%\setlength{\parskip}{0.7ex plus0.3ex minus0.3ex} % 段落间距
\allowdisplaybreaks[4]       %% 允许多行公式中间换页
%--------------------------------



%----------------------- 字体设置 ----------------------------------------
\usepackage{fontspec}
%\usepackage{xeCJK}             %--中文字体
\usepackage{newtxtext}       %-- 设置文字部分的英文字体为 Times New Roman
%\usepackage{newtxmath}    %-- 设置数学式中的英文字体为 Times New Roman
%-----------------------------------------------
%\setCJKmonofont{FZYTK.ttf}  %--方正姚体
%\newcommand{\fangzhengyaoti}{\CJKfamily{FZYTK.ttf}}
 \newcommand{\myfont}[1]{\setCJKfamilyfont{font}{#1}\CJKfamily{font}}


%\newcommand{\chuhao}{\fontsize{42pt}{\baselineskip}\selectfont}
\newcommand{\xiaochuhao}{\fontsize{36pt}{\baselineskip}\selectfont}
%\newcommand{\yihao}{\fontsize{28pt}{\baselineskip}\selectfont}
\newcommand{\xiaoyihao}{\fontsize{24pt}{\baselineskip}\selectfont}
\newcommand{\erhao}{\fontsize{21pt}{\baselineskip}\selectfont}
%\newcommand{\xiaoerhao}{\fontsize{18pt}{\baselineskip}\selectfont}
\newcommand{\sanhao}{\fontsize{15.75pt}{\baselineskip}\selectfont}
\newcommand{\sihao}{\fontsize{14pt}{\baselineskip}\selectfont}
\newcommand{\xiaosihao}{\fontsize{12pt}{\baselineskip}\selectfont}
\newcommand{\wuhao}{\fontsize{10.5pt}{\baselineskip}\selectfont}
%\newcommand{\xiaowuhao}{\fontsize{9pt}{\baselineskip}\selectfont}
%\newcommand{\liuhao}{\fontsize{7.875pt}{\baselineskip}\selectfont}
%\newcommand{\qihao}{\fontsize{5.25pt}{\baselineskip}\selectfont}
%-------------------------------------------------------------------------

%---------------- 设置目录、标题 -------------------------------------
\usepackage{titletoc,titlesec}

\renewcommand{\contentsname}{{
\begin{center}
\sihao\heiti 目\ \ 录
\end{center}
}}
\usepackage[titles]{tocloft}
\renewcommand\cftsecdotsep{\cftdotsep}
\renewcommand\cftsecleader{\cftdotfill{\cftsecdotsep}}

\newcommand{\sectionname}{节}
\titleformat{\section}[block]{\sihao\fangsong}{\thesection}{10pt}{}
\titleformat{\subsection}[block]{\xiaosihao\heiti}{\thesubsection}{10pt}{}
\titleformat{\subsubsection}[block]{\xiaosihao\fangsong}{\thesubsubsection}{10pt}{}
%-- 下面设置各级标题左边以及与上下文内容的间距 ---
\titlespacing{\section}{0pt}{0.6ex}{0.6ex}
\titlespacing{\subsection}{0.5ex}{0.3ex}{0.3ex}
\titlespacing{\subsubsection}{1ex}{0.3ex}{0.3ex}
%------------------------------------------------------------------------



%--------------- 设置定理环境 -----------------------------------------
\newtheoremstyle{DingLi1}
        {0.6ex plus 0.5ex minus .2ex}%上方的空行
        {0.6ex plus 0.5ex minus .2ex}%下方的空行
        {\kaishu}%内容字体
        {}%缩进
        {\heiti}%定理头部字体
        {.}%定理头部后的标点
        {1em}%定理头部后的空格
        {}%定理头部的说明
\theoremstyle{DingLi1}
\numberwithin{equation}{section}
\newtheorem{theorem}{\hspace{2em}定理}[section]
\newtheorem{definition}{\hskip 2em 定义}[section]
\newtheorem{lemma}{\hskip 2em 引理}[section]
\newtheorem{corollary}{\hskip 2em 推论}[section]
\newtheorem{assumption}{\hskip 2em 条件}[section]
\newtheorem{remark}{\hskip 2em 注}[section]
\newtheorem{proposition}{\hskip 2em 命题}[section]
\newtheorem{property}{\hskip 2em 性质}[section]

\newtheoremstyle{DingLi2}
        {1ex plus 0.5ex minus .2ex}%上方的空行
        {1ex plus 0.5ex minus .2ex}%下方的空行
        {\songti}%内容字体
        {}%缩进
        {\heiti\large\color{red} }%定理头部字体
        {.}%定理头部后的标点
        {1em}%定理头部后的空格
        {}%定理头部的说明
\theoremstyle{DingLi2}
\newtheorem{example}{\hskip 2em 问题}[section]
%---------------------------------------------------------------------

%---------- 设置证明环境 ---------------------------------------------
\makeatletter% amsthm: get rid of \itshape and \@addpunct{.}
\renewenvironment{proof}[1][\proofname]{\par%
\pushQED{\qed}%
\normalfont \topsep6\p@\@plus6\p@\relax%
\trivlist%
\item[\hskip\labelsep%
#1]\ignorespaces%
}{%
\popQED\endtrivlist\@endpefalse%
}
\makeatother
\renewcommand{\proofname}{\heiti\large\color{blue} 证明}%注意字体的设置
%---------------------------------------------------------------------

%-------------- 自定义命令 ------------------------------
\def\d{\mathop{}\!\mathrm{d}} %-- 一阶微分符号d
\def\l{\lambda}\def\L{\Lambda}
\def\D{\Delta}
\def\de{\delta}
\def\gm{\gamma}
\def\a{\alpha}
\def\b{\beta}
\def\div{{\rm div}}
%---------------------------------------------------------

\begin{document}



\title{ 泛函分析作业}
\author{}
\date{}
\maketitle

\section{第2周}

\begin{example}
设$P_r[a,b]$是定义在闭区间$[a,b]$上的所有\textbf{有理系数多项式函数}的全体. 显然, $(P_r[a,b],d)$是连续函数空间$(C[a,b],d)$的距离子空间, 其中
   $$d(f,g)=\max_{t\in [a,b]}\left|f(t)-g(t)\right|,\quad \forall f,g\in C[a,b].$$
证明: $P_r[a,b]$是$C[a,b]$的可数稠密子集, 从而$C[a,b]$可分.
\end{example}

\begin{proof}

 Step1. 对任意$n\in \Bbb{N}$, 设$P_r^n[a,b]$是定义在$[a,b]$上的所有\textbf{有理系数$n$次多项式函数}的全体, 则$P_r^n[a,b]$是可数集. 由于
$$P_r[a,b]=\bigcup_{n=0}^\infty P_r^n[a,b],$$
则$P_r[a,b]$也是可数集.

Step2. 下证$P_r[a,b]$按距离$d$在$P[a,b]$中稠密.

    任取$h\in P[a,b]$,
    $$h(t)=a_0+a_1 t+a_2 t^2+\cdots+a_n t^n,\quad t\in[a,b],$$
    其中$a_0,a_1,\cdots,a_n\in \Bbb{R}$, $n\in \Bbb{N}$. 令
    $$M=\max_{1\leq k\leq n}\max_{t\in [a,b]}|t|^k>0.$$
    根据有理数集$\Bbb{Q}$在实数集$\Bbb{R}$中的稠密性, 对任意$\epsilon>0$, 存在$q_0,q_1,\cdots,q_n\in\Bbb{Q}$使得
    $$\left|a_0-q_0\right|<\frac{1}{n+1}\epsilon,\quad \left|a_1-q_1\right|<\frac{1}{(n+1)M}\epsilon,\quad\cdots,\quad\left|a_n-q_n\right|<\frac{1}{(n+1)M}\epsilon.$$
    令
    $$g(t)=q_0+q_1t+\cdots +q_n t^n,\quad t\in [a,b],$$
    则$g\in P_r[a,b]$, 并且对任意$t\in [a,b]$都有
    $$
    \begin{array}{rcl}
    &&|h(t)-g(t)|\\
    &\leq&\left|a_0-q_0\right|+|a_1-q_1|\cdot|t|+\cdots+|a_n-q_n|\cdot |t|^n\\
    &\leq&\left|a_0-q_0\right|+|a_1-q_1|M+\cdots+|a_n-q_n|M\\
    &<&\epsilon,
    \end{array}
    $$
    从而
    $$\max_{t\in[a,b]}|h(t)-g(t)|<\epsilon.$$
    综上, 对任意$h\in P[a,b]$以及任意$\epsilon>0$, 存在$g\in P_r[a,b]$使得
    $$d(h,g)=\max_{t\in [a,b]}|h(t)-g(t)|<\epsilon,$$
    所以$P_r[a,b]$按距离$d$在$P[a,b]$中稠密.

Step3. 根据Weierstrauss逼近定理, $P[a,b]$按距离$d$在$C[a,b]$中稠密, 则对任意$\epsilon>0$以及任意$f\in C[a,b]$, 存在$h\in P[a,b]$使得
$$d(f,h)<\frac12 \epsilon.$$
由Step2可知, 存在$g\in P_r[a,b]$使得
$$d(h,g)<\frac12 \epsilon,$$
从而$d(f,g)\leq d(f,h)+d(h,g)<\epsilon$.

综上, $P_r[a,b]$是$C[a,b]$的可数稠密子集, 从而$C[a,b]$可分.
\end{proof}


\begin{example}
按以下步骤证明
   \begin{theorem}[Riemann-Lebesgue引理] 设$f\in L[a,b]$, 对应的Fourier系数为
   $$a_n=\int_a^b f(x)\sin nx {\rm d}x,\quad  b_n=\int_a^b f(x)\cos nx {\rm d}x,\quad n\in \Bbb{N},$$
    则$a_n,\,b_n\to 0 \quad (n\to\infty)$.
\end{theorem}
Step1. 若$f$是$[a,b]$上的简单函数(P80定义3), 证明上述结论成立.

Step2. 设$S[a,b]$是定义在闭区间$[a,b]$上的简单函数的全体. 显然, $S[a,b]$是$L[a,b]$的距离子空间, 其中距离
       $$d(f,g)=\int_a^b |f(t)-g(t)|{\rm d}t,\quad \forall f,g\in L[a,b].$$
    证明: $S[a,b]$是$L[a,b]$的稠密子集.

Step3. 利用稠密性, 证明Riemann-Lebesgue引理成立.
\end{example}

\begin{proof}

Step0. 设$h$是$[a,b]$上的一个阶梯函数,
$$
h(x)
=\left\{\begin{array}{ll}
c_1,&x\in (a_1,b_1),\\
c_2,&x\in(a_2,b_2),\\
\cdots&\cdots\\
c_k,&x\in (a_k,b_k),\\
0,&x\in [a,b]\setminus\cup_{i=1}^k(a_i,b_i),
\end{array}\right.
$$
其中$c_1,c_2,\cdots,c_k$为常数, $(a_1,b_1),\cdots,(a_k,b_k)$是$[a,b]$中互不相交的非空开子区间. 于是,
$$
\begin{aligned}
&\int_a^b h(x)\sin nx dx\\
=&\sum_{i=1}^k c_i\int_{a_i}^{b_i}\sin nx dx\\
=&\frac{1}{n}\sum_{i=1}^k c_i(\cos na_i-\cos n b_i)\\
\to&0\quad(n\to\infty).
\end{aligned}
$$
同理可证
$$\int_a^b h(x)\cos nx dx\to 0\quad (n\to \infty).$$

Step1.
设$E$是$[a,b]$中的可测子集, $\chi$是$E$的特征函数, 即
$$
\chi(x)
=\left\{\begin{array}{ll}
1,&x\in E,\\
0,&x\in[a,b]\setminus E,
\end{array}\right.
$$
下证
$$\int_a^b \chi(x)\sin nx dx\to 0\quad (n\to \infty).$$

令$\tilde{E}=E\cap (a,b)$, 则$\tilde{E}$也可测并且$m(E\setminus\tilde{E})=0$. **对任意$\epsilon>0$**, 存在开集$G\subset [a,b]$使得$\tilde{E}\subset G$并且
$$m(G\setminus \tilde{E})<\frac12 \epsilon.$$
另一方面, 根据$\Bbb{R}^1$中开集的构造定理(P44), $G$可表为
$$G=\bigcup_{i=1}^\infty O_i,$$
其中$O_i=(a_i,b_i)$是$G$的构成区间, 从而
$$\sum_{i=1}^\infty (b_i-a_i)=mG\leq b-a<+\infty.$$
于是, 对上述$\epsilon>0$, 存在$N\in \Bbb{N}_+$使得
$$\sum_{i=N+1}^\infty (b_i-a_i)<\frac{1}{2}\epsilon.$$
令$V=\cup_{i=1}^N (a_i, b_i)$, 并定义阶梯函数
$$
h(x)
=\left\{\begin{array}{ll}
1,&x\in V,\\
0,&x\in[a,b]\setminus V,
\end{array}\right.
$$
则
$$
\begin{aligned}
&\int_a^b |\chi(x)-h(x)|dx\\
=&\left(\int_{E\setminus V}+\int_{V\setminus E}+\int_{[a,b]\setminus(E\cup V)}\right)|\chi(x)-h(x)|dx\\
=&\left(\int_{\tilde{E}\setminus V}+\int_{V\setminus \tilde{E}}+\int_{[a,b]\setminus(\tilde{E}\cup V)}\right)|\chi(x)-h(x)|dx\\
=&\int_{\tilde{E}\setminus V} |1-0|dx +\int_{V\setminus \tilde{E}}|0-1|dx+\int_{[a,b]\setminus(\tilde{E}\cup V)}|0-0|dx\\
=&m(\tilde{E}\setminus V)+m(V\setminus \tilde{E})\\
\leq&m(G\setminus V)+m(G\setminus \tilde{E})\\
<&\frac12\epsilon+\frac12\epsilon=\epsilon,
\end{aligned}
$$
从而
$$
\begin{aligned}
0\leq&\left|\int_a^b \chi(x)\sin nx dx\right|\\
\leq&\left|\int_a^b \big(\chi(x)-h(x)\big)\sin nx dx\right|+\left|\int_a^b h(x)\sin nx dx\right|\\
\leq&\int_a^b \left|\chi(x)-h(x)\right||\sin nx| dx+\left|\int_a^b h(x)\sin nx dx\right|\\
\leq&\int_a^b \left|\chi(x)-h(x)\right| dx+\left|\int_a^b h(x)\sin nx dx\right|\\
<&\epsilon+\left|\int_a^b h(x)\sin nx dx\right|.
\end{aligned}
$$
由于$h$是阶梯函数, 综合Step0, 数列极限的迫敛性以及$\epsilon>0$的任意性, 可得
$$\int_a^b \chi(x)\sin nx dx\to 0\quad (n\to \infty).$$

设$f$是$[a,b]$上的简单函数,
$$f(x)=\sum_{i=1}^k c_i \chi_i(x),$$
其中
\begin{itemize}
   \item[(i)] $[a,b]=\cup_{i=1}^k E_i$, $E_1,E_2,\cdots,E_k$是$[a,b]$中互不相交的可测子集;
    \item[(ii)] $c_1,c_2,\cdots,c_k$是非负常数;
    \item[(iii)] $\chi_i(x)$是$E_i$的特征函数, 即
    $$
    \chi_i(x)
    =\left\{\begin{array}{ll}
    1,&x\in E_i,\\
    0,&x\in[a,b]\setminus E_i.
    \end{array}\right.
    $$
\end{itemize}
由前面的结论可知
$$\int_a^b f(x)\sin nx dx=\sum_{i=1}^k c_i\int_a^b \chi_i(x)\sin nx dx\to 0\quad (n\to\infty). $$

同理可证
$$\int_a^b f(x)\cos nx dx\to 0\quad (n\to\infty). $$

Step2. (P118) 设$f\in L[a,b]$, 则$f^+$和$f^-$也是$[a,b]$上的非负L可积函数, 从而, 根据非负可测函数L积分的定义(P102, 定义1), 对任意$\epsilon>0$, 存在$[a,b]$上的简单函数$\phi_1$, $\phi_2$, 使得
$$0\leq \phi_1(x)\leq f^+(x),\quad 0\leq \phi_2(x)\leq f^-(x),\quad \forall x\in [a,b],$$
并且
$$\int_a^b f^+(x)dx-\frac{\epsilon}{2}\leq\int_a^b \phi_1(x) dx\leq \int_a^b f^+(x)dx,$$
$$\int_a^b f^-(x)dx-\frac{\epsilon}{2}\leq\int_a^b \phi_2(x) dx\leq \int_a^b f^-(x)dx.$$
令$\phi(x)=\phi_1(x)-\phi_2(x)$, 则$\phi$也是$[a,b]$上的简单函数, 并且
$$
\begin{aligned}
d(f,\phi)=&\int_a^b |f(x)-\phi(x)| dx\\
=&\int_a^b |f^+(x)-f^-(x)-\phi_1(x)+\phi_2(x)|dx\\
\leq&\int_a^b|f^+(x)-\phi_1(x)|dx+\int_a^b |f^-(x)-\phi_2(x)|dx\\
<&\frac{\epsilon}{2}+\frac{\epsilon}{2}=\epsilon.
\end{aligned}
$$

综上, $S[a,b]$是$L[a,b]$的稠密子集.

Step3. 由Step2, 对任意$f\in L[a,b]$以及任意$\epsilon>0$, 存在$g\in S[a,b]$, 使得
$$d(f,g)=\int_a^b|f(x)-g(x)|dx<\epsilon,$$
从而
$$
\begin{aligned}
0\leq&\left|\int_a^b f(x)\sin nx dx\right|\\
\leq&\left|\int_a^b \big(f(x)-g(x)\big)\sin nx dx\right|+\left|\int_a^b g(x)\sin nx dx\right|\\
\leq&\int_a^b \left|f(x)-g(x)\right||\sin nx| dx+\left|\int_a^b g(x)\sin nx dx\right|\\
\leq&\int_a^b \left|f(x)-g(x)\right| dx+\left|\int_a^b g(x)\sin nx dx\right|\\
<&\epsilon+\left|\int_a^b g(x)\sin nx dx\right|.
\end{aligned}
$$
另一方面, 根据Step1, 就有
$$\int_a^b g(x)\sin nx dx \to 0\quad(n\to \infty).$$
综上, 由以上两式, 结合数列极限的迫敛性以及$\epsilon>0$的任意性, 可得
$$\int_a^b f(x)\sin nx dx\to 0\quad (n\to \infty).$$

同理可证
$$\int_a^b f(x)\cos nx dx\to 0\quad (n\to \infty).$$
\end{proof}

\end{document}
