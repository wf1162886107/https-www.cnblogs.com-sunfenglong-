\documentclass[UTF8,oneside,12pt]{article}  %--- 单面打印用oneside,双面打印就改为twoside
                                                                        %--- 12pt对应中文字号小四号
\usepackage{ctex}
%--- 可以用下面一命令代替上面两行 -----------
%\documentclass[UTF8,oneside,12pt]{ctexart}   %-------ctex + article=ctexart
%-----------------------------------------------


%------- 一些常用数学宏包 --------------------------
\usepackage{amsmath,amsthm}
\usepackage{amsfonts,amssymb}
\usepackage{mathrsfs}
\usepackage{bm}
\usepackage{dsfont}
%------------------------------------------------------
\usepackage[colorlinks,linkcolor=red,anchorcolor=blue,citecolor=red]{hyperref}  %--- 超链接宏包
\usepackage{cite}          %--- 引用宏包

\usepackage{indentfirst}   %-- 首行缩进宏包
\usepackage{array,tabularx}           %-- 列表宏包
\usepackage{diagbox}      %-- 改宏包可用于制做有斜线表头的表格
\usepackage{color}          %-- 颜色宏包
\usepackage{graphicx}    %--- 插图宏包

\usepackage{caption}      %--- 该宏包用来定制图表标题
\captionsetup{font={footnotesize,bf}}  %-- 图表标题设置为小五号黑体


%-------  geometry 宏包用来调整页面的页边距。默认的页面边距为A4标准。
%\usepackage{geometry}

%------  根据学校文件中的课程论文模板的实际情况,下面通过指定geometry 宏包的选项参数来自定义页面边距 ----------------
\usepackage[
  hoffset=5mm,%%装订线
  left=2.5cm,      %% 左边距
  right=2.5cm,     %% 右边距
  top=2.5cm,       %% 上边距 包括页眉
  bottom=2cm,    %% 下边距 包括页脚
  headheight=0cm, %%页眉
  footskip=1.0cm,  %% 页脚
  paperwidth=21cm,
  paperheight=29.7cm
  ]{geometry}

\def\baselinestretch{1}         %-- 行间距为1倍行间距
%\setlength{\baselineskip}{20pt}  %-- 行间距为固定值20磅
%\setlength{\parskip}{0.7ex plus0.3ex minus0.3ex} % 段落间距
\allowdisplaybreaks[4]       %% 允许多行公式中间换页
%--------------------------------



%----------------------- 字体设置 ----------------------------------------
\usepackage{fontspec}
%\usepackage{xeCJK}             %--中文字体
\usepackage{newtxtext}       %-- 设置文字部分的英文字体为 Times New Roman
%\usepackage{newtxmath}    %-- 设置数学式中的英文字体为 Times New Roman
%-----------------------------------------------
%\setCJKmonofont{FZYTK.ttf}  %--方正姚体
%\newcommand{\fangzhengyaoti}{\CJKfamily{FZYTK.ttf}}
 \newcommand{\myfont}[1]{\setCJKfamilyfont{font}{#1}\CJKfamily{font}}


%\newcommand{\chuhao}{\fontsize{42pt}{\baselineskip}\selectfont}
\newcommand{\xiaochuhao}{\fontsize{36pt}{\baselineskip}\selectfont}
%\newcommand{\yihao}{\fontsize{28pt}{\baselineskip}\selectfont}
\newcommand{\xiaoyihao}{\fontsize{24pt}{\baselineskip}\selectfont}
\newcommand{\erhao}{\fontsize{21pt}{\baselineskip}\selectfont}
%\newcommand{\xiaoerhao}{\fontsize{18pt}{\baselineskip}\selectfont}
\newcommand{\sanhao}{\fontsize{15.75pt}{\baselineskip}\selectfont}
\newcommand{\sihao}{\fontsize{14pt}{\baselineskip}\selectfont}
\newcommand{\xiaosihao}{\fontsize{12pt}{\baselineskip}\selectfont}
\newcommand{\wuhao}{\fontsize{10.5pt}{\baselineskip}\selectfont}
%\newcommand{\xiaowuhao}{\fontsize{9pt}{\baselineskip}\selectfont}
%\newcommand{\liuhao}{\fontsize{7.875pt}{\baselineskip}\selectfont}
%\newcommand{\qihao}{\fontsize{5.25pt}{\baselineskip}\selectfont}
%-------------------------------------------------------------------------

%---------------- 设置目录、标题 -------------------------------------
\usepackage{titletoc,titlesec}

\renewcommand{\contentsname}{{
\begin{center}
\sihao\heiti 目\ \ 录
\end{center}
}}
\usepackage[titles]{tocloft}
\renewcommand\cftsecdotsep{\cftdotsep}
\renewcommand\cftsecleader{\cftdotfill{\cftsecdotsep}}

\newcommand{\sectionname}{节}
\titleformat{\section}[block]{\sihao\fangsong}{\thesection}{10pt}{}
\titleformat{\subsection}[block]{\xiaosihao\heiti}{\thesubsection}{10pt}{}
\titleformat{\subsubsection}[block]{\xiaosihao\fangsong}{\thesubsubsection}{10pt}{}
%-- 下面设置各级标题左边以及与上下文内容的间距 ---
\titlespacing{\section}{0pt}{0.6ex}{0.6ex}
\titlespacing{\subsection}{0.5ex}{0.3ex}{0.3ex}
\titlespacing{\subsubsection}{1ex}{0.3ex}{0.3ex}
%------------------------------------------------------------------------



%--------------- 设置定理环境 -----------------------------------------
\newtheoremstyle{DingLi1}
        {0.6ex plus 0.5ex minus .2ex}%上方的空行
        {0.6ex plus 0.5ex minus .2ex}%下方的空行
        {\kaishu}%内容字体
        {}%缩进
        {\heiti}%定理头部字体
        {.}%定理头部后的标点
        {1em}%定理头部后的空格
        {}%定理头部的说明
\theoremstyle{DingLi1}
\numberwithin{equation}{section}
\newtheorem{theorem}{\hspace{2em}定理}[section]
\newtheorem{definition}{\hskip 2em 定义}[section]
\newtheorem{lemma}{\hskip 2em 引理}[section]
\newtheorem{corollary}{\hskip 2em 推论}[section]
\newtheorem{assumption}{\hskip 2em 条件}[section]
\newtheorem{remark}{\hskip 2em 注}[section]
\newtheorem{proposition}{\hskip 2em 命题}[section]
\newtheorem{property}{\hskip 2em 性质}[section]

\newtheoremstyle{DingLi2}
        {1ex plus 0.5ex minus .2ex}%上方的空行
        {1ex plus 0.5ex minus .2ex}%下方的空行
        {\songti}%内容字体
        {}%缩进
        {\heiti\large\color{red} }%定理头部字体
        {.}%定理头部后的标点
        {1em}%定理头部后的空格
        {}%定理头部的说明
\theoremstyle{DingLi2}
\newtheorem{example}{\hskip 2em 问题}[section]
%---------------------------------------------------------------------

%---------- 设置证明环境 ---------------------------------------------
\makeatletter% amsthm: get rid of \itshape and \@addpunct{.}
\renewenvironment{proof}[1][\proofname]{\par%
\pushQED{\qed}%
\normalfont \topsep6\p@\@plus6\p@\relax%
\trivlist%
\item[\hskip\labelsep%
#1]\ignorespaces%
}{%
\popQED\endtrivlist\@endpefalse%
}
\makeatother
\renewcommand{\proofname}{\heiti\large\color{blue} 证明}%注意字体的设置
%---------------------------------------------------------------------

%-------------- 自定义命令 ------------------------------
\def\d{\mathop{}\!\mathrm{d}} %-- 一阶微分符号d
\def\l{\lambda}\def\L{\Lambda}
\def\D{\Delta}
\def\de{\delta}
\def\gm{\gamma}
\def\a{\alpha}
\def\b{\beta}
\def\div{{\rm div}}
%---------------------------------------------------------

\begin{document}



\title{ 泛函分析作业}
\author{}
\date{}
\maketitle

\section{第3周}

\begin{example}
设$(X,d)$是度量空间, $\{x_n\}$是$(X,d)$中的Cauchy点列, 证明: $\{x_n\}$收敛当且仅当$\{x_n\}$存在收敛子列.
\end{example}

\begin{proof}
必要性是显然的.

下证充分性. 设Cauchy点列$\{x_n\}$存在收敛子列$\{x_{n_k}\}$使得$x_{n_k}\to x$ $(k\to \infty)$.任取$\epsilon>0$. 一方面, 由于$\{x_n\}$是Cauchy点列, 则存在$N=N(\epsilon)\in \Bbb{N}_+$, 使得
\begin{equation}\label{3.1-1}
d(x_n, x_m)<\frac12 \epsilon,\quad \forall m,n>N.
\end{equation}
另一方面, 由于$x_{n_k}\to x$ $(k\to \infty)$, 则存在$K=K(\epsilon)\in \Bbb{N}_+$使得
\begin{equation}\label{3.1-2}
n_k >N\quad\text{并且}\quad d(x_{n_k},x) <\frac12 \epsilon,\quad \forall k>K.
\end{equation}


综上, 由(\ref{3.1-1})-(\ref{3.1-2})式, 对任意$n>N$, 取$k=K+1$, 就有
\begin{eqnarray*}
d(x_n ,x)\leq d(x_n,x_{n_k})+d(x_{n_k}, x)<\frac12 \epsilon+\frac12 \epsilon=\epsilon,
\end{eqnarray*}
所以$x_n\to x$ $(n\to \infty)$.
\end{proof}

\begin{example}
设$f$是度量空间$(X,d)$到$\Bbb{R}$的连续映射, $M$是$X$中的紧集, 证明: 连续映射$f$在紧集$M$上能够取到最值, 即存在$x_0,x_1\in M$使得
$$f(x_0)=\min_{x\in M}f(x),\quad f(x_1)=\max_{x\in M}f(x).$$
\end{example}
\begin{proof}
Step1. 设
$$l=\inf_{x\in M}f(x).$$
下证$l\in \Bbb{R}$.

反证法, 假设$l=-\infty$, 则对任意$n\in \Bbb{N}_+$, 存在$x_n\in M$使得
\begin{equation*}
f(x_n)<-n,
\end{equation*}
于是
\begin{equation}\label{3.2-1}
f(x_n)\to -\infty\quad (n\to \infty).
\end{equation}
另一方面, 由于$\{x_n\}\subset M$并且$M$是紧集, 则存在$\{x_n\}$的子列$\{x_{n_k}\}$以及$x\in M$使得
$$x_{n_k}\to x\quad (k\to \infty).$$
根据映射$f$的连续性, 就有
$$f(x_{n_k})\to f(x)\in \Bbb{R}\quad (k\to \infty).$$
这与(\ref{3.2-1})式矛盾. 所以$l\in \Bbb{R}$.

Step2. 根据下确界的定义, 存在$\{x_n\}\subset M$(称为极小化序列)使得
$$f(x_n)\to l\quad (n\to \infty).$$
由于$M$是紧集, 则存在$\{x_n\}$的子列$\{x_{n_k}\}$以及$x\in M$使得
$$x_{n_k}\to x\quad (k\to \infty).$$
根据映射$f$的连续性, 就有
$$\inf_{x\in M}f(x)=l=\lim_{n\to \infty}f(x_n)=\lim_{k\to \infty}f(x_{n_k})=f\left(\lim_{k\to \infty} x_{n_k}\right)=f(x).$$
所以连续映射$f$在紧集$M$上可以取到最小值.

同理可证, 连续映射$f$在紧集$M$上可以取到最大值.
\end{proof}


\begin{example}
\begin{definition}[H\"{o}lder连续函数]设$\alpha\in (0,1]$. 若$f\in C[a,b]$满足
$$[f]_\alpha=\sup_{\substack{x,y\in[a,b],\\ x\neq y}}\frac{|f(x)-f(y)|}{|x-y|^\alpha}<+\infty,$$
则称$f$是$[a,b]$上具有指数$\alpha$的H\"{o}lder连续函数. $C[a,b]$中所有具有指数$\alpha$的H\"{o}lder连续函数的全体记为$C^{0,\alpha}[a,b]$.
\end{definition}
\begin{itemize}
  \item[(1)]令
    $$\bar{d}(f,g)=\max_{t\in [a,b]}|f(t)-g(t)|+[f-g]_{\alpha}, \quad\forall f,g\in C^{0,\alpha}[a,b], $$
    证明$(C^{0,\alpha}[a,b],\bar{d})$是一个度量空间.
  \item[(2)] 证明$(C^{0,\alpha}[a,b],\bar{d})$是完备的度量空间.
  \item[(3)] 利用Ascoli-Arezela定理证明, 若$M$是$(C^{0,\alpha}[a,b],\bar{d})$中的有界集, 则$M$是$(C[a,b],d)$中的列紧集, 其中$d$是最大值距离, 即
    $$d(f,g)=\max_{t\in [a,b]}|f(t)-g(t)|,\quad \forall f,g\in C[a,b].$$
\end{itemize}
\end{example}
\begin{proof}
(1) 任取$f,g\in C^{0,\alpha}[a,b]$, 对任意$x,y\in [a,b]$且$x\neq y$, 都有
\begin{eqnarray*}
&&\frac{|(f-g)(x)-(f-g)(y)|}{|x-y|^\alpha} \\
&\leq& \frac{|f(x)-f(y)|}{|x-y|^\alpha}+\frac{|g(x)-g(y)|}{|x-y|^\alpha}\\
&\leq &[f]_{\alpha}+[g]_{\alpha}<+\infty,
\end{eqnarray*}
从而
$$[f-g]_{\alpha}=\sup_{\substack{x,y\in[a,b],\\ x\neq y}}\frac{|(f-g)(x)-(f-g)(y)|}{|x-y|^\alpha}\leq [f]_{\alpha}+[g]_{\alpha}<+\infty.$$
所以$\bar{d}(f,g)$的定义是合理的.
\begin{itemize}
  \item[(i)]显然$\bar{d}(f,g)\geq 0$. 由于$d(f,g)\leq \bar{d}(f,g)$, 根据$d(f,g)$的正定性可知, $\bar{d}(f,g)=0$等价于
      $$f(t)=g(t),\quad \forall t\in [a,b],$$
      从而等价于$f=g$.
      \item[(ii)]设$f,g,h\in C^{0,\alpha}[a,b]$, 则
      $$d(f,g)\leq d(f,h)+d(h,g).$$
      另一方面, 对任意$x,y\in [a,b]$且$x\neq y$, 都有
      \begin{eqnarray*}
      &&\frac{|(f-g)(x)-(f-g)(y)|}{|x-y|^\alpha}\\
      &=&\frac{\big|[(f-h)+(h-g)](x)-[(f-h)+(h-g)](y)\big|}{|x-y|^\alpha}\\
      &\leq&\frac{|(f-h)(x)-(f-h)(y)|}{|x-y|^\alpha}+\frac{|(h-g)(x)-(h-g)(y)|}{|x-y|^\alpha},
      \end{eqnarray*}
      从而
      $$[f-g]_{\alpha}\leq [f-h]_{\alpha}+[h-g]_{\alpha}.$$
      综上,
      $$\bar{d}(f,g)\leq \bar{d}(f,g)+\bar{d}(h,g).$$
\end{itemize}
所以$(C^{0,\alpha}[a,b],\bar{d})$是一个度量空间.

(2) 设$\{f_n\}$是空间$(C^{0,\alpha}[a,b],\bar{d})$中的Cauchy点列. 由于$C^{0,\alpha}[a,b]\subset C[a,b]$并且
$$0\leq d(f,g)\leq  \bar{d}(f,g), \quad\forall f,g\in C^{0,\alpha}[a,b],$$
易证$\{f_n\}$也是$(C[a,b],d)$中的Cauchy点列. 根据$(C[a,b],d)$的完备性, 存在$f\in C[a,b]$使得
$$d(f_n,f)\to 0\quad (n\to \infty).$$
下证$f\in C^{0,\alpha}[a,b]$并且
$$\bar{d}(f_n,f)\to 0\quad (n\to \infty).$$

由于$\{f_n\}$是空间$(C^{0,\alpha}[a,b],\bar{d})$中的Cauchy点列, 从而是$(C^{0,\alpha}[a,b],\bar{d})$中的有界点列(p219第14题), 于是存在$M>0$使得对任意$x,y\in [a,b]$并且$x\neq y$都有
\begin{equation}\label{3.3-1}
\frac{|f_n(x)-f_n(y)|}{|x-y|^\alpha}\leq [f_n]_\alpha =[f_n-0]_\alpha \leq \bar{d}(f_n,0)\leq M,\quad \forall n\in \Bbb{N}_+.
\end{equation}
由于函数列$\{f_n\}$在$[a,b]$上一致收敛于$f$, 那么也逐点收敛于$f$, 即对任意$x\in [a,b]$, 都有
\begin{equation}\label{3.3-2}
f_n(x)\to f(x)\quad (n\to \infty).
\end{equation}
因此, 在(\ref{3.3-1})两端令$n\to \infty$, 可得
$$\frac{|f(x)-f(y)|}{|x-y|^\alpha}\leq M,\quad \forall x,y\in [a,b],\,x\neq y,$$
从而$[f]_\alpha <+\infty$, $f\in C^{0,\alpha}[a,b]$.

对任意$\epsilon>0$, 由于$\{f_n\}$是空间$(C^{0,\alpha}[a,b],\bar{d})$中的Cauchy点列, 则存在正整数$N$, 使得对任意$m,n>N$, 都有
\begin{equation*}
\frac{\big|[f_n(x)-f_m(x)]-[f_n(y)-f_m(y)]\big|}{|x-y|^\alpha}\leq [f_n-f_m]_\alpha \leq \bar{d}(f_n,f_m)<\epsilon,\quad \forall x,y\in [a,b],\,x\neq y.
\end{equation*}
在上式中固定$x,y$以及$n>N$, 令$m\to\infty$, 结合(\ref{3.3-2})式可得
\begin{equation*}
\frac{\big|[f_n(x)-f(x)]-[f_n(y)-f(y)]\big|}{|x-y|^\alpha}\leq \epsilon,\quad \forall n\geq N,\ \forall x,y\in [a,b],\,x\neq y,
\end{equation*}
所以
\begin{equation*}
  [f_n-f]_\alpha\leq \epsilon, \forall n>N.
\end{equation*}
综上
$$[f_n-f]_\alpha\to 0\quad (n\to \infty),$$
从而
$$\bar{d}(f_n,f)=d(f_n,f)+[f_n-f]_\alpha\to 0\quad(n\to \alpha).$$
所以$(C^{0,\alpha}[a,b],\bar{d})$是完备的度量空间.



(3) 设$M$在$(C^{0,\alpha}[a,b],\bar{d})$中有界. 由于$C^{0,\alpha}[a,b]\subset C[a,b]$并且
$$0\leq d(f,g)\leq  \bar{d}(f,g), \quad\forall f,g\in C^{0,\alpha}[a,b],$$
所以$M$也在$(C[a,b],d)$中有界. 任取$\{f_n\}\subset M$, 则$\{f_n\}$既是$(C^{0,\alpha}[a,b],\bar{d})$中的有界点列, 又是$(C[a,b],d)$中的有界点列, 从而函数列$\{f_n\}$在$[a,b]$上一致有界. 下证函数列$\{f_n\}$在$[a,b]$上
等度连续.

由于$\{f_n\}$是$(C^{0,\alpha}[a,b],\bar{d})$中的有界点列, 则存在$M>0$, 使得
\begin{equation*}
[f_n]_\alpha =[f_n-0]_\alpha \leq \bar{d}(f_n,0)\leq M.
\end{equation*}
从而
\begin{equation}\label{3.3-3}
|f_n(x)-f_n(y)|\leq M |x-y|^\alpha,\quad \forall n\in \Bbb{N}_+,\ \forall x,y\in [a,b].
\end{equation}
对任意$\epsilon>0$, 取
$$\delta=\left(\frac{\epsilon}{M}\right)^{\frac{1}{\alpha}}>0,$$
则对任意$x,y\in [a,b]$且$|x-y|<\delta$, 根据$\alpha\in (0,1]$以及\eqref{3.3-3}式可得
$$|f_n(x)-f_n(y)|\leq M |x-y|^\alpha<M\delta^\alpha=\epsilon,$$
因此函数列$\{f_n\}$在$[a,b]$上等度连续.

根据Ascoli-Arezela定理, 点列$\{f_n\}$在空间$(C[a,b],d)$中有收敛子列, 由此可知集合$M$是空间$(C[a,b],d)$中的列紧集.\\
\textcolor[rgb]{1.00,0.00,0.00}{由于$\{f_n\}$是$(C^{0,\alpha}[a,b],\bar{d})$中的有界点列, 根据(2)的证明的前半部分可知, 上述收敛子列$\{f_{n_k}\}$的极限$f$也在$C^{0,\alpha}[a,b]$中. 然而, 虽然$\{f_{n_k}\}$在$(C[a,b],d)$中收敛, 但是却不能保证$\{f_{n_k}\}$是$(C^{0,\alpha}[a,b],\bar{d})$)的Cauchy点列, 因此我们无法像(2)的证明的后半部分那样证明
$$\left[f_{n_k}-f\right]_\alpha \to 0\quad (k\to \infty).$$}
\end{proof}


\end{document}
