\documentclass[UTF8,oneside,12pt]{article}  %--- 单面打印用oneside,双面打印就改为twoside
                                                                        %--- 12pt对应中文字号小四号
\usepackage{ctex}
%--- 可以用下面一命令代替上面两行 -----------
%\documentclass[UTF8,oneside,12pt]{ctexart}   %-------ctex + article=ctexart
%-----------------------------------------------


%------- 一些常用数学宏包 --------------------------
\usepackage{amsmath,amsthm}
\usepackage{amsfonts,amssymb}
\usepackage{mathrsfs}
\usepackage{bm}
\usepackage{dsfont}
%------------------------------------------------------
\usepackage[colorlinks,linkcolor=red,anchorcolor=blue,citecolor=red]{hyperref}  %--- 超链接宏包
\usepackage{cite}          %--- 引用宏包

\usepackage{indentfirst}   %-- 首行缩进宏包
\usepackage{array,tabularx}           %-- 列表宏包
\usepackage{diagbox}      %-- 改宏包可用于制做有斜线表头的表格
\usepackage{color}          %-- 颜色宏包
\usepackage{graphicx}    %--- 插图宏包

\usepackage{caption}      %--- 该宏包用来定制图表标题
\captionsetup{font={footnotesize,bf}}  %-- 图表标题设置为小五号黑体


%-------  geometry 宏包用来调整页面的页边距。默认的页面边距为A4标准。
%\usepackage{geometry}

%------  根据学校文件中的课程论文模板的实际情况,下面通过指定geometry 宏包的选项参数来自定义页面边距 ----------------
\usepackage[
  hoffset=5mm,%%装订线
  left=2.5cm,      %% 左边距
  right=2.5cm,     %% 右边距
  top=2.5cm,       %% 上边距 包括页眉
  bottom=2cm,    %% 下边距 包括页脚
  headheight=0cm, %%页眉
  footskip=1.0cm,  %% 页脚
  paperwidth=21cm,
  paperheight=29.7cm
  ]{geometry}

\def\baselinestretch{1}         %-- 行间距为1倍行间距
%\setlength{\baselineskip}{20pt}  %-- 行间距为固定值20磅
%\setlength{\parskip}{0.7ex plus0.3ex minus0.3ex} % 段落间距
\allowdisplaybreaks[4]       %% 允许多行公式中间换页
%--------------------------------



%----------------------- 字体设置 ----------------------------------------
\usepackage{fontspec}
%\usepackage{xeCJK}             %--中文字体
\usepackage{newtxtext}       %-- 设置文字部分的英文字体为 Times New Roman
%\usepackage{newtxmath}    %-- 设置数学式中的英文字体为 Times New Roman
%-----------------------------------------------
%\setCJKmonofont{FZYTK.ttf}  %--方正姚体
%\newcommand{\fangzhengyaoti}{\CJKfamily{FZYTK.ttf}}
 \newcommand{\myfont}[1]{\setCJKfamilyfont{font}{#1}\CJKfamily{font}}


%\newcommand{\chuhao}{\fontsize{42pt}{\baselineskip}\selectfont}
\newcommand{\xiaochuhao}{\fontsize{36pt}{\baselineskip}\selectfont}
%\newcommand{\yihao}{\fontsize{28pt}{\baselineskip}\selectfont}
\newcommand{\xiaoyihao}{\fontsize{24pt}{\baselineskip}\selectfont}
\newcommand{\erhao}{\fontsize{21pt}{\baselineskip}\selectfont}
%\newcommand{\xiaoerhao}{\fontsize{18pt}{\baselineskip}\selectfont}
\newcommand{\sanhao}{\fontsize{15.75pt}{\baselineskip}\selectfont}
\newcommand{\sihao}{\fontsize{14pt}{\baselineskip}\selectfont}
\newcommand{\xiaosihao}{\fontsize{12pt}{\baselineskip}\selectfont}
\newcommand{\wuhao}{\fontsize{10.5pt}{\baselineskip}\selectfont}
%\newcommand{\xiaowuhao}{\fontsize{9pt}{\baselineskip}\selectfont}
%\newcommand{\liuhao}{\fontsize{7.875pt}{\baselineskip}\selectfont}
%\newcommand{\qihao}{\fontsize{5.25pt}{\baselineskip}\selectfont}
%-------------------------------------------------------------------------

%---------------- 设置目录、标题 -------------------------------------
\usepackage{titletoc,titlesec}

\renewcommand{\contentsname}{{
\begin{center}
\sihao\heiti 目\ \ 录
\end{center}
}}
\usepackage[titles]{tocloft}
\renewcommand\cftsecdotsep{\cftdotsep}
\renewcommand\cftsecleader{\cftdotfill{\cftsecdotsep}}

\newcommand{\sectionname}{节}
\titleformat{\section}[block]{\sihao\fangsong}{\thesection}{10pt}{}
\titleformat{\subsection}[block]{\xiaosihao\heiti}{\thesubsection}{10pt}{}
\titleformat{\subsubsection}[block]{\xiaosihao\fangsong}{\thesubsubsection}{10pt}{}
%-- 下面设置各级标题左边以及与上下文内容的间距 ---
\titlespacing{\section}{0pt}{0.6ex}{0.6ex}
\titlespacing{\subsection}{0.5ex}{0.3ex}{0.3ex}
\titlespacing{\subsubsection}{1ex}{0.3ex}{0.3ex}
%------------------------------------------------------------------------



%--------------- 设置定理环境 -----------------------------------------
\newtheoremstyle{DingLi1}
        {0.6ex plus 0.5ex minus .2ex}%上方的空行
        {0.6ex plus 0.5ex minus .2ex}%下方的空行
        {\kaishu}%内容字体
        {}%缩进
        {\heiti}%定理头部字体
        {.}%定理头部后的标点
        {1em}%定理头部后的空格
        {}%定理头部的说明
\theoremstyle{DingLi1}
\numberwithin{equation}{section}
\newtheorem{theorem}{\hspace{2em}定理}[section]
\newtheorem{definition}{\hskip 2em 定义}[section]
\newtheorem{lemma}{\hskip 2em 引理}[section]
\newtheorem{corollary}{\hskip 2em 推论}[section]
\newtheorem{assumption}{\hskip 2em 条件}[section]
\newtheorem{remark}{\hskip 2em 注}[section]
\newtheorem{proposition}{\hskip 2em 命题}[section]
\newtheorem{property}{\hskip 2em 性质}[section]

\newtheoremstyle{DingLi2}
        {1ex plus 0.5ex minus .2ex}%上方的空行
        {1ex plus 0.5ex minus .2ex}%下方的空行
        {\songti}%内容字体
        {}%缩进
        {\heiti\large\color{red} }%定理头部字体
        {.}%定理头部后的标点
        {1em}%定理头部后的空格
        {}%定理头部的说明
\theoremstyle{DingLi2}
\newtheorem{example}{\hskip 2em 问题}[section]
%---------------------------------------------------------------------

%---------- 设置证明环境 ---------------------------------------------
\makeatletter% amsthm: get rid of \itshape and \@addpunct{.}
\renewenvironment{proof}[1][\proofname]{\par%
\pushQED{\qed}%
\normalfont \topsep6\p@\@plus6\p@\relax%
\trivlist%
\item[\hskip\labelsep%
#1]\ignorespaces%
}{%
\popQED\endtrivlist\@endpefalse%
}
\makeatother
\renewcommand{\proofname}{\heiti\large\color{blue} 证明}%注意字体的设置
%---------------------------------------------------------------------

%-------------- 自定义命令 ------------------------------
\def\d{\mathop{}\!\mathrm{d}} %-- 一阶微分符号d
\def\l{\lambda}\def\L{\Lambda}
\def\D{\Delta}
\def\de{\delta}
\def\gm{\gamma}
\def\a{\alpha}
\def\b{\beta}
\def\div{{\rm div}}
%---------------------------------------------------------

\begin{document}



\title{ 泛函分析作业}
\author{}
\date{}
\maketitle

\section{第1周}

\begin{definition}[等价距离] 设集合$X$上有两种距离:$d_1$, $d_2$. 如果$X$中按距离$d_1$收敛的点列$\{x_n\}$都在距离$d_2$下收敛于同一点, 并且按距离$d_2$收敛的点列$\{x_n\}$都在距离$d_1$下收敛于同一点, 即
$$d_1(x_n,x)\to 0 \Longleftrightarrow d_2(x_n,x)\to 0,$$
则称距离$d_1$和$d_2$等价.
\end{definition}

\begin{example}
设$d(x,y)$是集合$X$上的距离, 令
    $$\tilde{d}(x,y)=\frac{d(x,y)}{1+d(x,y)}.$$
    证明: $\tilde{d}(x,y)$也是$X$上的距离, 并且$\tilde{d}$与$d$等价.
\end{example}
\begin{proof}
显然, 对任意$x,y\in X$,
    $$\tilde{d}(x,y)=\frac{d(x,y)}{1+d(x,y)}\in \Bbb{R}.$$
    \begin{itemize}
      \item[(i)]由距离$d(x,y)$的正定性可知$\tilde{d}(x,y)\geq 0$, 并且$\tilde{d}(x,y)=\frac{d(x,y)}{1+d(x,y)}=0$等价于$d(x,y)=0$, 进而等价于$x=y$.
      \item[(ii)]由距离$d(x,y)$ 的三点不等式可知,对任意$x,y,z\in X$, 总有
          $$d(x,y)\leq d(x,z)+d(y,z),$$
          从而, 根据函数
    $$f(t)=\frac{t}{1+t},\quad t\in [0,+\infty)$$
    的单调递增性, 就有
    $$
    \begin{array}{rcl}
    \tilde{d}(x,y)=\frac{d(x,y)}{1+d(x,y)}&\leq&\frac{d(x,z)+d(y,z)}{1+d(x,z)+d(y,z)}\\
    &=&\frac{d(x,z)}{1+d(x,z)+d(y,z)}+\frac{d(y,z)}{1+d(x,z)+d(y,z)}\\
    &\leq&\frac{d(x,z)}{1+d(x,z)}+\frac{d(y,z)}{1+d(y,z)}\\
    &=&\tilde{d}(x,z) +\tilde{d}(y,z).
    \end{array}
    $$
    \end{itemize}
综上, $\tilde{d}(x,y)$也是空间$X$上的距离.

注意到
    $$0\leq \tilde{d}(x,y)=\frac{d(x,y)}{1+d(x,y)}<1,$$
于是
\begin{equation}\label{w1-1}
d(x,y) =\frac{\tilde{d}(x,y)}{1-\tilde{d}(x,y)}.
\end{equation}
若点列$\{x_n\}\subset X$和点$x\in X$满足
    $$d(x_n,x)\to 0 \quad (n\to \infty),$$
则根据数列极限的四则运算法则, 就有
    $$\tilde{d}(x_n,x)=\frac{d(x_n,x)}{1+d(x_n,x)}\to 0\quad (n\to\infty).$$
若点列$\{x_n\}\subset X$和点$x\in X$满足
    $$\tilde{d}(x_n,x)\to 0 \quad (n\to \infty),$$
同样根据(\ref{w1-1})式以及数列极限的四则运算法则, 就有
    $${d}(x_n,x)=\frac{\tilde{d}(x_n,x)}{1-\tilde{d}(x_n,x)}\to 0\quad (n\to\infty).$$
所以距离$d$和$\tilde{d}$等价.
\end{proof}

\begin{remark}上述距离空间$(X,\tilde{d})$中任何两点的距离都小于1, 从而任何子集都是有界集. 上述结论说明, 任何距离空间上$(X,d)$上都能够找到与$d$等价的"有界"距离$\tilde{d}$.
\end{remark}


\begin{example}
在$\Bbb{R}^N$中可定义两种距离:
    $$d_1(x,y)=\sqrt{\sum_{i=1}^N \left|\xi_i-\eta_i\right|^2},$$
    $$d_2(x,y)=\max_{1\leq i\leq N}\left|\xi_i-\eta_i\right|,$$
    其中$x=(\xi_1,\xi_2,\cdots,\xi_N)\in \Bbb{R}^N,$ $y=(\eta_1,\eta_2,\cdots,\eta_N)\in \Bbb{R}^N.$ 证明:$d_1$和$d_2$等价.
\end{example}


\begin{proof}
对任意$x=(\xi_1,\xi_2,\cdots,\xi_N)\in \Bbb{R}^N,$ $y=(\eta_1,\eta_2,\cdots,\eta_N)\in \Bbb{R}^N,$ 都有
    $$\max_{1\leq i\leq N} |\xi_i-\eta_i|^2\leq \sum_{i=1}^N |\xi_i-\eta_i|^2 \leq N\max_{1\leq i\leq N}|\xi_i-\eta_i|^2,$$
    从而
    $$\max_{1\leq i\leq N} |\xi_i-\eta_i|\leq \sqrt{\sum_{i=1}^N |\xi_i-\eta_i|^2}\leq \sqrt{N} \max_{1\leq i\leq N} |\xi_i-\eta_i|,$$
   即
    $$d_2(x,y)\leq d_1(x,y)\leq \sqrt{N} d_2(x,y).\label{w1-2}$$
若点列$\{x_n\}\subset \Bbb{R}^N$和点$x\in \Bbb{R}^N$满足
    $$d_1(x_n,x)\to 0 \quad (n\to \infty),$$
由(\ref{w1-2})式的前半部分以及数列极限的迫敛性可知
$$d_2(x_n,x)\to 0 \quad (n\to \infty).$$
若点列$\{x_n\}\subset \Bbb{R}^N$和点$x\in \Bbb{R}^N$满足
    $$d_2(x_n,x)\to 0 \quad (n\to \infty),$$
由(\ref{w1-2})式的后半部分以及数列极限的迫敛性可知
$$d_1(x_n,x)\to 0 \quad (n\to \infty).$$
综上, $d_1$和$d_2$等价.
\end{proof}


\begin{remark}
 若距离空间$X$上的两种距离$d_1$和$d_2$满足
    $$C_1 d_1(x,y)\leq d_2(x,y)\leq C_2 d_1(x,y),\quad \forall x,y\in X,$$
    其中$C_1,\,C_2>0$是正的常数, 则$d_1$与$d_2$一定等价.
\end{remark}

\end{document}
